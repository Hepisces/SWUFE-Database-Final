\begin{CJK*}{UTF8}{song}

我们围绕SQL语句的验证与分析需求,设计并实现了一款集SQL执行、语句验证、优化建议生成与结果可视化于一体的智能分析工具——Awesql。该工具以命令行交互为核心,面向SQLite数据库环境,构建了从查询输入到图形输出的完整处理链路。通过设计和集成多个智能工具,Awesql显著增强了对SQL查询语义的理解与纠错能力,实现了从静态语法验证到动态语义适配与性能评估的多维升级。

Awesql系统采用高度模块化、可组合的架构设计,不仅便于功能扩展与维护,也兼顾了专业开发人员的控制需求与非专业用户的易用性,在数据库交互智能化、自动化与人机协同方面提供了切实可行的技术路径,是数据库开发工具在智能化方向的一次尝试。具体而言,我们的Awesql在以下方面实现了创新:

(1)端到端的智能分析流程

Awesql打通了从自然语言输入到结构化SQL执行再到数据可视化的完整闭环,构建了统一的交互通道。在架构层面,系统支持SQL语句与自然语言两类输入,并根据任务类型动态调用Text2SQL模块与SQL Checker模块。此外,系统输出不仅包含查询结果,还自动生成符合数据特征的可视化图表,并支持数据库结构的ER图展示,提升了整体数据分析效率与用户认知体验。

(2)智能驱动的模块工具设计与集成

我们构建和接入了Text2SQL工具、基于Agent的SQL checker验证器以及自适应的可视化查询工具。其中SQL checker通过上下文注入与三阶段思维链,系统可在理解用户意图的基础上完成对语法、语义、执行性能及代码规范性的深度分析,最终生成结构化、解释性强的优化建议。

尽管Awesql在功能集成与智能化方面取得了一定成效,但在当前实现中仍存在一些不足。一是当前系统仅适配SQLite数据库,未实现对MySQL、PostgreSQL等主流数据库管理系统及其方言的支持,限制了其在异构数据库环境中的适用性和迁移能力。二是Text2SQL功能依赖于本地部署的大型语言模型,要求较高的计算资源,降低了其在低资源终端或轻量级环境中的可访问性。此外,当前的SQL分析与优化建议主要面向结构较为清晰的查询任务,对于包含嵌套、动态构造或复杂逻辑的SQL语句,系统在语义解析与可视化推荐方面的适应性仍显不足,优化效果与解释能力存在波动。

为进一步增强Awesql系统的适用范围与智能能力,针对这些不足,我们的后续工作将从系统与智能两方面推进:

(1)系统层面

①扩展数据库兼容性:我们希望将当前架构扩展至支持MySQL、PostgreSQL等主流数据库,通过方言适配与执行引擎封装,提升系统的通用性与生态兼容性。


②构建图形化界面或IDE插件:考虑到命令行的使用门槛,我们考虑将Awesql功能封装为VS Code等主流IDE的插件,降低工具的使用难度,拓展用户群体。


③增强可视化交互能力:在现有可视化模块基础上,引入更多可配置参数与图表类型,允许用户进行自定义分析与图形展示,同时支持多维度联动与动态更新。


(2)智能化层面


①Text2SQL优化:结合用户需求与数据库结构,通过更精细的提示工程优化Text2SQL模型,提升在复杂查询中的表现,确保生成的SQL语句更贴合真实需求。

②验证优化:在SQL checker部分,用户输入或查询结果后,系统根据查询的实际执行环,自动选择合适的优化策略,确保生成的SQL语句在不同数据库上都有优异的执行性能。

③一键式操作:现有的Awesql模块支持灵活调用,但考虑工具实际的测试和应用场景,可以加入一键式操作,一次性自动调用Awesql从数据导入到可视化的全流程,降低操作门槛,提高使用便捷性。


\end{CJK*}