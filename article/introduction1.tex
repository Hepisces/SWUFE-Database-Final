\begin{CJK*}{UTF8}{song}
随着信息技术的迅猛发展和数据规模的持续扩大,数据库及数据库管理系统(Database Management System, DBMS)已成为现代信息系统不可或缺的核心组件。数据库技术广泛应用于金融、电商、医疗、政务、科研、司法等诸多领域,承担着数据存储、管理、查询与分析的重要职责。数据库技术能方便地维护数据、更严密地控制数据和更有效地利用数据,数据库的数据存储具有冗余度小、易共享和独立性强等特点[1]。

关系型数据库管理系统(Relational Database Management System, RDBMS)凭借其成熟的事务管理机制、灵活的结构化查询语言SQL(Structured Query Language, SQL)以及良好的数据一致性保障,成为目前应用最为广泛的数据库类型之一。关系型数据库是以多个表格的形式进行表达,并将数据存储在表格中的数据库系统[2],每个表格以每一组行和列的数据组成。关系型数据库遵循结构化查询语言的标准,使用SQL来执行数据的增删和改查的操作,SQL作为一种强大的查询语言,它使得用户能够灵活地查询、增加、更新和删除数据。关系型数据库的优点在于数据结构非常清晰,容易理解且易于维护,具有事务一致性。这样的优势让它适用于需要严格保证数据完整性的应用场景。

数据库发展迅速并被广泛应用的当下,对于需要快速上手的非专业用户,数据库的使用仍存在一定的门槛。首先,数据库环境的搭建与配置通常较为繁琐,对于非专业用户而言存在技术障碍。其次,关系型数据库依赖SQL语言进行数据操作与查询,尽管SQL是一种声明式语言,即关注于实现目的而非实现过程,但快速学习与掌握SQL语法规则、查询流程、优化手段对于初学者同样存在学习成本。此外,不管是命令行窗口还是数据库管理工具,其都缺乏可视化直观反馈,体现在:查询流程往往隐藏于后台,用户难以及时理解查询逻辑及数据流动路径,不利于排查问题与优化查询;执行结果往往以表格形式展示,难以直观表达数据蕴含的丰富信息。

为此,本研究专注于设计一种集成化、智能化、交互友好、可视化的数据库命令行工具,具备数据库快速初始化、关系模式便捷录入、SQL语句智能纠错、自然语言查询、查询流程与结果可视化等功能,将极大降低数据库应用门槛,提升数据交互效率和使用体验,具有重要的理论价值与实践意义。

\end{CJK*}