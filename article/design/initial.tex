\begin{CJK*}{UTF8}{song}

此模块负责为首次运行系统准备必要的数据库环境。用户在命令行执行 \textit{awesql import-data} 命令,该命令有两个可选参数,分别为\textit{--der}和\textit{—db-name},分别指定数据文件夹路径和数据库名称,默认参数分别为Smart\_Home\_DATA 和project2025。

命令输入后,系统首先调用 db.db\_exists() 检查预定义的数据库文件(如 project2025.db)是否存在且包含数据。若已存在,则提示用户并终止操作,防止意外覆盖;若数据库不存在,系统会查找指定数据目录下的 DDL.sql 文件,通过 db.create\_tables() 函数,读取该文件内容并利用 sqlite3 的 executescript() 方法一次性执行所有 CREATE TABLE 语句,完成数据库表结构的构建。

表结构创建成功后, 系统通过 db. import\_real\_data() 函数,按预设的并且符合外键依赖关系的顺序(例如$ customer.sql \rightarrow devlist.sql \rightarrow ...$),逐一读取数据目录下的 .sql 文件。对每个文件中的每一行 INSERT 语句进行处理和执行,将数据填充到相应的表中。

同时,为了后续操作的需要,在首次初始化时,db.create\_default\_config\_if\_not\_exists() 会自动创建 awesql\_config.json 配置文件,并将 DDL.sql 的路径记录其中,供后续模块(如Text2SQL)使用。

为了设计简便,系统强制要求创建表结构的文件命名为DDL.sql,并与所有导入数据的文件存放在同一文件夹,导入数据文件的名称也需要与表名称一致。


\end{CJK*}