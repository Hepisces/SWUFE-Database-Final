\begin{CJK*}{UTF8}{song}

此模块负责执行SQL查询并以多种形式呈现其过程与结果。无论是用户直接通过 run命令输入的SQL,还是由 ask命令生成的SQL,都可以被传递到 db.execute\_query() 函数。此函数对输入SQL执行两个操作:1)通过\textit{EXPLAIN QUERY PLAN SQL}: 获取数据库引擎为该查询制定的执行计划;2)通过SQL: 执行查询本身,获取结果数据集。

这两个操作的结果分别被读入两个独立的pandas DataFrame:plan\_df 和 result\_df。其中 plan\_df 被传递给visualizer.draw\_query\_plan ()。该函数将plan\_df重建为树状结构,并匹配预设的解释文案,最后利用 rich 库在控制台渲染出带注释的彩色执行计划树。result\_df 则被用于两种可视化:1)visualizer .print\_results\_table() ,将结果的前10行渲染成一个格式化的控制台表格,供用户快速预览和精确读数。2)visualizer.visualize\_query\_result() ,启动一个交互式会话,引导用户从查询结果的列中选择用于图表 x 轴、y 轴和分类(hue)的数据。根据用户的选择和指定的图表类型(如柱状图、折线图),系统利用 matplotlib 和 seaborn 库生成图表,并将其保存为图像文件,同时自动尝试打开该文件。

为了避免潜在风险,系统会通过正则话自动检测输入的SQL查询,并只允许SELECT查询通过,其他所有与修改和删除相关的命令将被自动禁止并返回警告。

\end{CJK*}