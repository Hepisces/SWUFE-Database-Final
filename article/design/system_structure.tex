\begin{CJK*}{UTF8}{song}

我们设计的命令行工具采用分层架构, 将功能按逻辑划分为三个主要层:表现层、业务逻辑层和数据访问层,每一层通过不同的组件实现。

{\begin{CJK*}{UTF8}{zhhei}\subsubsection{表现层}\end{CJK*}}

表现层通过cli.py实现,作为唯一入口,负责与用户直接交互。这一层通过typer和rich库完成命令行的交互式,可以通过解析用户输入的命令(如 init, ask, run, reset)和参数,调用对应实现逻辑,并将最终结果(文本、表格、图表路径)格式化后呈现给用户。
{\begin{CJK*}{UTF8}{zhhei}\subsubsection{业务逻辑层}\end{CJK*}}

业务逻辑层通过 Text2SQL.py, checker.py, visualizer.py 分别实现,处理所有核心任务。Text2SQL.py可以将用户的自然语言问题转换为结构化查询语言(SQL),并结合数据库关系模式作为提示词与LLM进行交互。checker.py 模块作为一个关键的验证器,负责对用户输入或AI生成的SQL查询进行语法和基础语义的合法性检查,确保执行的查询是安全和有效的。visualizer.py 模块负责所有可视化任务,将抽象的查询计划(Query Plan)和查询结果(Result Set)数据,分别转换为人类易于理解的树状图和多种统计图表(如柱状图、折线图等)。

{\begin{CJK*}{UTF8}{zhhei}\subsubsection{数据访问层}\end{CJK*}}

数据访问层通过db.py 实现,封装了所有与底层数据库SQLite的交互细节,所有上层模块都通过此层访问数据库,实现包括但不仅限于数据库连接、表结构创建(DDL执行)、数据导入、查询执行(包括 \textit{EXPLAIN QUERY PLAN})以及数据库重置等功能。
值得一提的是,无论是从数据库获取的查询结果,还是查询计划,都被统一抽象为DataFrame。这可以极大地简化在不同模块(如 db, checker, visualizer)之间的数据传递和处理流程。

\end{CJK*}